%\documentclass{...}
%modification history
% 9 jul 2011
% 29 nov 2011

\documentclass[11pt, letterpaper]{article}
\usepackage[margin=1in]{geometry}

% \documentclass[11pt, letterpaper]{article}
%
% % ----- margins -----
%
% \topmargin -1.5cm         % read Lamport p.163
% \oddsidemargin -0.04cm    % read Lamport p.163
% \evensidemargin -0.04cm   % same as oddsidemargin but for left-hand pages
%
% % ----- texts -----
%
% \textwidth 16.59cm
% \textheight 21.94cm
%
% % ----- indendts and spacing -----
%
% \parskip 0pt            	% spacing between paragraphs
% %\renewcommand{\baselinestretch}{1.5}	% uncomment for 1.5 spacing
%
% \parindent 7mm		      % leading space for paragraphs between lines
%
% % ----- page # -----
%
% %\pagestyle{empty}         % uncomment if don't want page numbers





\usepackage{amsfonts, amsmath, amssymb, amsthm}
\usepackage{comment} 
\usepackage{graphicx}
\usepackage{ifthen}
\usepackage{latexsym}
%\usepackage{times}


%=== use the these packages if they are not already in use ===
\usepackage{amsfonts, amsmath, amssymb, amsthm}
\usepackage[nocompress]{cite}
\usepackage{color}
\usepackage{graphicx}
\usepackage{microtype}
\usepackage[normalem]{ulem}
\usepackage{wrapfig}
%=============================================================

%===== fonts =====
\def\ttt{\texttt}
\def\tsc{\textsc}

%===== spacing =====

\def\extraspacing{\vspace{3mm} \noindent}
\def\figcapup{\vspace{-1mm}}
\def\figcapdown{\vspace{-0mm}}
\def\hgap{\textrm{\hspace{1mm}}}
\def\thmvgap{\vspace{0mm}}
\def\vgap{\vspace{1mm}}


%===== tabbing =====

\def\tab{\hspace{3mm}}
\def\tabpos{\hspace{4mm} \= \hspace{4mm} \= \hspace{4mm} \= \hspace{4mm} \=
\hspace{4mm} \= \hspace{4mm} \= \hspace{4mm} \= \hspace{4mm} \= \hspace{4mm}
\kill}
\newcommand{\mytab}[1]{\begin{tabbing}\tabpos #1\end{tabbing}}

%===== blocks =====

\newtheorem{theorem}{Theorem}
\newtheorem{lemma}{Lemma}
\newtheorem{corollary}{Corollary}
\newtheorem{proposition}{Proposition}
\newtheorem{definition}{Definition}
\newtheorem{problem}{Problem}

\newcommand{\boxminipg}[2]{\begin{center}\fbox{\begin{minipage}{#1}#2\end{minipage}}\end{center}}
\newcommand{\minipg}[2]{\begin{center}\begin{minipage}{#1}#2\end{minipage}\end{center}}
\newcommand{\myitems}[1]{\begin{itemize}#1\end{itemize}}
\newcommand{\myenums}[1]{\begin{enumerate}#1\end{enumerate}}
\newcommand{\myfig}[1]{\begin{figure}\centering #1\end{figure}}
\newcommand{\myfigstar}[1]{\begin{figure*}\centering #1\end{figure*}}

%===== math macros =====

\newcommand{\bm}[1]{\textrm{\boldmath${#1}$}}
% \newcommand{\smat}[2]{\left[\begin{tabular}{#1}#2\end{tabular}\right]}
% \newcommand{\bmat}[2]{\left|\begin{tabular}{#1}#2\end{tabular}\right|}
\newcommand{\bmat}[1]{\begin{bmatrix}#1\end{bmatrix}}
\newcommand{\vmat}[1]{\begin{vmatrix}#1\end{vmatrix}}
\newcommand{\myeqn}[1]{\begin{eqnarray}#1\end{eqnarray}}

\newcommand{\myset}[1]{\{#1\}}

%\def\bm{\boldmath}
%\def\defeq{\stackrel{\textrm{\tiny{def}}}{=}}
\def\mit{\mathit}
\def\defeq{:=}
\def\eps{\epsilon}
\def\fr{\frac}
\def\-{\mbox{-}}
\def\real{\mathbb{R}}

\def\tO{\tilde{O}}

\def\lc{\lceil}
\def\lf{\lfloor}
\def\rc{\rceil}
\def\rf{\rfloor}

\def\nn{\nonumber}

\def\Pr{\mathbf{Pr}}
\def\expt{\mathbf{E}}
\def\var{\mathbf{var}}

\def\dcl{\{\!\!\{}
\def\dcr{\}\!\!\}}
\def\bigdcl{\Big\{\!\!\Big\{}
\def\bigdcr{\Big\}\!\!\Big\}}
\def\bigmid{\textrm{ $\Big|$ }}

\def\*{\star}

\DeclareMathOperator*{\argmin}{arg\,min}
\DeclareMathOperator*{\polylg}{polylg}
\DeclareMathOperator*{\polylog}{polylog}
\DeclareMathOperator*{\intr}{\cap}


%===== misc =====

\def\done{\qed \vspace{2mm}}	% end of proof
\def\tbc{\hspace*{\fill} $\textrm{{\em (to be continued)}}\blacktriangle$ \vspace{2mm}}
%\def\done{\hspace*{\fill} $\Box$}	% end of proof

%===== coloring =====

\newcommand{\red}[1]{\textcolor{red}{#1}}

\def\A{\mathcal{A}}
\def\C{\EuScript{C}}
\def\S{\EuScript{S}}
\def\dist{\mathit{dist}}

\def\A{\mathcal{A}}

\newboolean{solver}\setboolean{solver}{false}
%\newboolean{solver}\setboolean{solver}{true}
\ifthenelse{\boolean{solver}}{\includecomment{sol}}{\excludecomment{sol}}

\def\extraspacing{\vspace{5mm} \noindent}


\begin{document}

\extraspacing {\bf Problem 1.} Consider the merging problem in our PRAM discussion. Let $A_1$ be the array $(1, 17, 28, 29, 55, 61, 69, 80)$ and $A_2$ be the array $(10, 13, 25, 33, 38, 56, 72, 75)$. Give the content of array $B_1$.


\begin{sol} 
    \extraspacing {\bf Solution.} $B_1 = (1, 4, 6, 7, 10, 12, 13, 16)$.
\end{sol}

\extraspacing {\bf Problem 2.} Consider the sorting problem in EM. Let $A$ be the input file of $n$ integers (which is stored in $O(n/B)$ blocks). Give an algorithm to produce $O(n/M)$ files satisfying all the following requirements: 
\myitems{
    \item Each file stores at most $M$ integers of $A$ in ascending order using $O(M/B)$ blocks. 
    \item All the files are mutually disjoint. 
    \item The union of all the files is the set of integers in $A$.
}
Your algorithm must terminate in $O(n/B)$ I/Os. 

\begin{sol}
    \extraspacing {\bf Solution.} Load the first $M$ numbers of $A$ into memory, sort them, and then write the sorted list into a file of $O(M/B)$ blocks. Then, do the same to the next $M$ numbers of $A$. Repeat until having exhausted $A$. 
\end{sol}

\extraspacing {\bf Problem 3.} This question concerns the PRAM model. Suppose that we have already obtained a sorting algorithm $\A$ finishing in $f(n)$ steps when the number $p$ of CPUs equals $n$ (recall that $n$ is the number of integers to sort). Consider now the scenario where $p < n$. Describe how to use $\A$ to design an algorithm that finishes in $O(\fr{n}{p} \cdot f(n))$ steps.

\begin{sol}
    \extraspacing {\bf Solution.} The key is to simulate a step of $\A$, which runs on $n$ CPUs, by performing $O(n/p)$ steps with $p$ CPUs. I suggest giving a full mark as long as the student manages to make the above observation.

    \vgap

    For a complete answer, however, we must clarify the details of the reduction. Let $\Upsilon_1, \Upsilon_2, ..., \Upsilon_n$ be the $n$ CPUs of $\A$; call them the {\em $\A$-CPUs}. Define the {\em state} of each $\Upsilon_i$ ($i \in [1, n]$) as the current set of register values in $\Upsilon_i$. To simulate $\A$, we preserve the state of every $\Upsilon_i$ in $O(1)$ special memory words.

    \vgap

    We simulate every step of $\A$ --- call this an {\em $\A$-step} --- in $O(n/p)$ steps through $\lc n / p \rc$ phases. The $i$-th ($i \in [1, n]$) phase executes the atomic operations performed by $\Upsilon_{i \cdot p+1}, \Upsilon_{i \cdot p+2}, ..., \Upsilon_{(i+1) \cdot p}$ during the $\A$-step. For this purpose, we first perform $O(1)$ steps to load the state of $\Upsilon_j$, for each $j \in [i \cdot p+1, (i+1) \cdot p]$, into the $(j \mod p)$-th CPU. The atomic operations of the $p$ CPUs can then be carried out in another step. Because $\A$ is a CREW algorithm, no two CPUs can have a conflict in memory accesses during an $\A$-step. Therefore, we guarantee that the states of all $\A$-CPUs after   the $\lc n / p \rc$ phases are identical to those after the $\A$-step.
\end{sol}

\extraspacing {\bf Problem 4.} Give a PRAM algorithm that settles the sorting problem in $O(\fr{n}{p} \log^2 n)$ steps.

\begin{sol}
    \extraspacing {\bf Solution.} Apply the reduction in Problem 3 to our discussion in the seminar.
\end{sol}


\end{document}
