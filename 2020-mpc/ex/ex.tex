%\documentclass{...}

%modification history
% 9 jul 2011
% 29 nov 2011

\documentclass[11pt, letterpaper]{article}
\usepackage[margin=1in]{geometry}

% \documentclass[11pt, letterpaper]{article}
%
% % ----- margins -----
%
% \topmargin -1.5cm         % read Lamport p.163
% \oddsidemargin -0.04cm    % read Lamport p.163
% \evensidemargin -0.04cm   % same as oddsidemargin but for left-hand pages
%
% % ----- texts -----
%
% \textwidth 16.59cm
% \textheight 21.94cm
%
% % ----- indendts and spacing -----
%
% \parskip 0pt            	% spacing between paragraphs
% %\renewcommand{\baselinestretch}{1.5}	% uncomment for 1.5 spacing
%
% \parindent 7mm		      % leading space for paragraphs between lines
%
% % ----- page # -----
%
% %\pagestyle{empty}         % uncomment if don't want page numbers





\usepackage{amsfonts, amsmath, amssymb, amsthm}
\usepackage{comment} 
\usepackage{graphicx}
\usepackage{ifthen}
\usepackage{latexsym}
%\usepackage{times}


%=== use the these packages if they are not already in use ===
\usepackage{amsfonts, amsmath, amssymb, amsthm}
\usepackage[nocompress]{cite}
\usepackage{color}
\usepackage{graphicx}
\usepackage{microtype}
\usepackage[normalem]{ulem}
\usepackage{wrapfig}
%=============================================================

%===== fonts =====
\def\ttt{\texttt}
\def\tsc{\textsc}

%===== spacing =====

\def\extraspacing{\vspace{3mm} \noindent}
\def\figcapup{\vspace{-1mm}}
\def\figcapdown{\vspace{-0mm}}
\def\hgap{\textrm{\hspace{1mm}}}
\def\thmvgap{\vspace{0mm}}
\def\vgap{\vspace{1mm}}


%===== tabbing =====

\def\tab{\hspace{3mm}}
\def\tabpos{\hspace{4mm} \= \hspace{4mm} \= \hspace{4mm} \= \hspace{4mm} \=
\hspace{4mm} \= \hspace{4mm} \= \hspace{4mm} \= \hspace{4mm} \= \hspace{4mm}
\kill}
\newcommand{\mytab}[1]{\begin{tabbing}\tabpos #1\end{tabbing}}

%===== blocks =====

\newtheorem{theorem}{Theorem}
\newtheorem{lemma}{Lemma}
\newtheorem{corollary}{Corollary}
\newtheorem{proposition}{Proposition}
\newtheorem{definition}{Definition}
\newtheorem{problem}{Problem}

\newcommand{\boxminipg}[2]{\begin{center}\fbox{\begin{minipage}{#1}#2\end{minipage}}\end{center}}
\newcommand{\minipg}[2]{\begin{center}\begin{minipage}{#1}#2\end{minipage}\end{center}}
\newcommand{\myitems}[1]{\begin{itemize}#1\end{itemize}}
\newcommand{\myenums}[1]{\begin{enumerate}#1\end{enumerate}}
\newcommand{\myfig}[1]{\begin{figure}\centering #1\end{figure}}
\newcommand{\myfigstar}[1]{\begin{figure*}\centering #1\end{figure*}}

%===== math macros =====

\newcommand{\bm}[1]{\textrm{\boldmath${#1}$}}
% \newcommand{\smat}[2]{\left[\begin{tabular}{#1}#2\end{tabular}\right]}
% \newcommand{\bmat}[2]{\left|\begin{tabular}{#1}#2\end{tabular}\right|}
\newcommand{\bmat}[1]{\begin{bmatrix}#1\end{bmatrix}}
\newcommand{\vmat}[1]{\begin{vmatrix}#1\end{vmatrix}}
\newcommand{\myeqn}[1]{\begin{eqnarray}#1\end{eqnarray}}

\newcommand{\myset}[1]{\{#1\}}

%\def\bm{\boldmath}
%\def\defeq{\stackrel{\textrm{\tiny{def}}}{=}}
\def\mit{\mathit}
\def\defeq{:=}
\def\eps{\epsilon}
\def\fr{\frac}
\def\-{\mbox{-}}
\def\real{\mathbb{R}}

\def\tO{\tilde{O}}

\def\lc{\lceil}
\def\lf{\lfloor}
\def\rc{\rceil}
\def\rf{\rfloor}

\def\nn{\nonumber}

\def\Pr{\mathbf{Pr}}
\def\expt{\mathbf{E}}
\def\var{\mathbf{var}}

\def\dcl{\{\!\!\{}
\def\dcr{\}\!\!\}}
\def\bigdcl{\Big\{\!\!\Big\{}
\def\bigdcr{\Big\}\!\!\Big\}}
\def\bigmid{\textrm{ $\Big|$ }}

\def\*{\star}

\DeclareMathOperator*{\argmin}{arg\,min}
\DeclareMathOperator*{\polylg}{polylg}
\DeclareMathOperator*{\polylog}{polylog}
\DeclareMathOperator*{\intr}{\cap}


%===== misc =====

\def\done{\qed \vspace{2mm}}	% end of proof
\def\tbc{\hspace*{\fill} $\textrm{{\em (to be continued)}}\blacktriangle$ \vspace{2mm}}
%\def\done{\hspace*{\fill} $\Box$}	% end of proof

%===== coloring =====

\newcommand{\red}[1]{\textcolor{red}{#1}}


\newboolean{solver}\setboolean{solver}{false}
%\newboolean{solver}\setboolean{solver}{true}
\ifthenelse{\boolean{solver}}{\includecomment{sol}}{\excludecomment{sol}}


\begin{document}

\section*{Exercises}

\extraspacing {\bf Problem 1.} Which of the following can be a property of a minimal sorting algorithm? 


\vgap

\noindent A.\ It performs $O(\log n)$ supersteps, where $n$ is the number of elements to sort. \\
\noindent B.\ It requires a machine to send $O(n/p^{0.99})$ words in some superstep. \\
\noindent C.\ It requires a machine to spend $O((n/p) \log (n/p))$ CPU time in some superstep. \\
\noindent D.\ It requires a machine to use $O(n/p^{0.99})$ space in some superstep. 

\begin{sol}
   \extraspacing {\bf Solution.} C.
\end{sol}


\extraspacing {\bf Problem 2.} In the seminar, we introduced a minimal algorithm for sorting. Assuming that each machine has $n/p$ elements in its local storage at the beginning of the algorithm, answer the following questions: 
\myitems{
    \item [(a)] In Phase 1, how many elements are sampled from each machine in expectation? 
    \item [(b)] Still in Phase 1, how many elements does each machine {\em receive} in expectation?
}

\begin{sol}
   \extraspacing {\bf Solution.} (a) Since each element is sampled with probability $(p/n) \ln(np)$, the expected number of elements sampled is $(n/p) \cdot (p/n) \ln(np) = \ln(np)$. 
   
   \vgap 
   
   (b) Each machine receives the sample elements from all other machines. Therefore, it receives $(p-1) \ln(np)$ elements in expectation. 
\end{sol}

\extraspacing {\bf Problem 3.} In our argument for proving the lower bound on the load of the cartesian product problem, we had the sentence: ``Machine 1 sees $n + L$ elements overall $\Rightarrow$ it can produce at most $(\fr{n+L}{2})^2$ pairs.'' Give a proof of the sentence.

\begin{sol}
   \extraspacing {\bf Solution.} Let $x$ and $y$ be the number of red and blue elements Machine 1 sees, respectively. Thus, $x + y \le n + L$. The number of pairs that can be produced by Machine 1 is $xy$. We have: 
   \myeqn{
        \sqrt{xy} &\le& (x+y)/2 \hspace{10mm} \Rightarrow \nn \\
        xy &\le& ((x+y)/2)^2 \hspace{10mm} \Rightarrow \nn \\ 
        &=& \left(\fr{n+L}{2}\right)^2. \hspace{10mm} \nn 
   }
\end{sol}
\end{document}
