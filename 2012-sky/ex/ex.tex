\input{../def/yf-formatting}

\usepackage{amsfonts, amsmath, amssymb, amsthm}
\usepackage{graphicx}

\input{../def/yf-math-def}

\begin{document}

\noindent These are the questions from Yufei's lecture.

\extraspacing {\bf Question 1 (5\%).} Given two points $o, o'$, let us define that $o$ {\em dominates} $o'$ if the coordinate of $o$ is smaller than that of $o'$ on each dimension. By this definition, what is the skyline of the dataset shown in the figure below?

\begin{center}
    \includegraphics[height=60mm]{./artwork/ds.eps}
\end{center}

\noindent {\bf Answer:} $\{o_1, o_2, o_8\}$.

\extraspacing {\bf Question 2 (5\%).} Recall that the {\em SFS} algorithm works by first sorting all the data points according to a scoring function $f(x, y)$. Let the function be $f(x, y) = x$. For example, a point $(5, 4)$ has score 5. In other words, {\em SFS} processes the data points in ascending order of their scores. For each point $o$, its processing requires comparing $o$ to some other points. In the dataset shown in the above figure, when point $o_3$ is being processed, which point or points is $o_3$ compared to?

\extraspacing {\bf Answer:} $\{o_1, o_2\}$.

\end{document}
